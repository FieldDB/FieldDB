\documentclass[letterpaper, 12pt, dvips]{mitwpl}
\usepackage{pstricks}
\usepackage[novbox]{pdfsync}
\setlength{\headheight}{14.5pt}

%%% BEGIN DOCUMENT
\begin{document}

\papertitle{iCampo: Un aplicaci�n de colecci�n de datos ling��sticos}

\firsttitle{\LARGE iCampo: \\[.2ex]\Large   Un aplicaci\'on de colecci\'on de datos ling\"u\'isticos}
\runningtitle{iCampo/LingSync}

\author{MaryEllen Cathcart$^{a,b}$\\ Gina Cook$^b$\\ Theresa Deering$^{b,c}$\\ Yuliya Manyakina$^{d,e}$\\Gretchen McCulloch$^d$\\Hisako Noguchi$^e$}
\runningauthor{Cathcart et. al.}
\institution{$^a$University of Delaware $^b$iLanguage Lab $^c$Visit Scotland $^d$McGill University$^e$Concordia University}
\ackfn{ LingSync: A Free Tool for Creating and Maintaining a Shared Database For Communities, Linguists and Language Learners\\We would like to thank everybody who has contributed to this work.}

\maketitle

\section*{Spanish Abstract - ME} 

\section*{English Abstract}


LingSync is an OpenSource database system that allows language researchers to securely enter, store, organize, annotate, and share linguistic data. The application is accessible on any device; as it runs in a HTML5 browser, it runs on laptops (Mac  10.5 and above, Linux, Windows, ChromeBooks) as well as on mobile devices (Android and iPhone/iPad).  It is suitable for both online and offline use. %\footnote{Running offline requires  a local database. As of August 2012 it would be possible to use the app offline in Chrome Browsers (Mac, Linux \& Windows), ChromeBooks and on Android tablets, in the coming months/years other browsers/systems will support offline databases. Safari and Firefox may also work by the end of 2012.}  
Furthermore, the application is created with collaborative goals in mind;  data is syncable and sharable with other researchers.  Researchers can form teams that contribute to a single corpus, where team members use the application to modify and discuss the data. The system also has a simple and friendly user interface, allowing users to drag and drop data (audio, video, text), or record audio/video directly into the database. In addition, the application  has import and export capabilities for multiple file types. LingSync is designed from the ground up to conform to E-MELD and DataOne data management best practices, an important requirement for any database which will house data funded by granting agencies.  Most importantly, the application is designed intuitively and theory free, so it is not necessary to be a field linguist or programmer to figure out how it works. LingSync is hosted on cloud servers so that users can use it without knowing how to set up its servers, but also has an installation guide for linguistics department server administrators so that they can set up unlimited data usage on their own department servers.




%LingSync is a collaboration between  the programming fieldlinguists of iLanguage Lab LTD, Montreal, the Mig'maq Research Group at McGill University and the Prosody Lab at McGill University, Montreal Canada.

\section{Introduction}

\subsection{Why LingSync was created}
\subsubsection{The need - GRETCHEN}

LingSync was conceived out of the needs of language researchers doing
fieldwork or other large scale data collection. Linguistic fieldwork often requires researchers
to travel to places where a stable connection to the internet is not guaranteed.
Also, it often involves a group of researchers contributing to building a single database.
An ideal linguistic database should therefore work both online and offline as well as
making it easy to share and integrate data.

\subsubsection{Other programs: pros and cons - GRETCHEN}



There are several existing programs used for linguistic fieldwork; however, none of them fully satisfies the needs of field linguists for robust, collaborative, multi-platform data annotation and organization, both online and offline. For example, there are web-based databases which allow collaboration, such as the Online Linguistic Database (OLD), %URL/reference
 Karuk Dictionary and Texts,  % http://linguistics.berkeley.edu/~karuk/links.php 
and The Washo Project % http://washo.uchicago.edu/dictionary/dictionary.php
but they only work online, making them unusable for researchers looking to enter new data or search the database while in the field with limited or no internet access.
There are also non-web-based software programs such as Toolbox %http://www.sil.org/computing/toolbox/
and FLEx/FieldWorks, %http://fieldworks.sil.org/flex/
which are excellent for annotating
data and organizing data into various formats (corpus, grammar or lexicon).
However, with these offline tools, researchers each enter data on a single computer, making them more vulnerable to technical difficulties, and meaning that they must use a single device for all work on the language and cannot easily combine their data with others who work on related projects. Moreover, these tools run only on a single platform (either PC or Linux, but not both and not Mac or mobile devices). 
Another offline tool is general purpose database software such as FileMaker Pro, which can be customized for the purpose of language research. However, this incurs the same problems as other offline tools, while additionally often requiring that a programmer be hired to customize the software for the purpose of linguistic research. 

%The existing linguistic database programs, although useful, have various shortfalls that would hinder collection and integration of data. Some of them are constrained by the internet accessibility or by the computer platform types. Some others demand extra human work in order to integrate data. Data entry can require hours of a researcher�s time. 
All of the linguistic database programs surveyed did not provide a good user experience.
The number of clicks required and the delay between actions did not meet current
software engineering best practices. In addition to core functionalities, a good user
experience is necessary to ensure quality data management. LingSync grows
out of discussion with a number of fieldworkers dissatisfied with currently available
options.


\subsubsection{Technological background: why we can make this now - GINA} 

\subsection{Design principles - HISAKO} 
\subsubsection{Open (open source, open access)}
\subsubsection{Standards-compliant (Unicode, EMELD, Leipzig, XML)}
\subsubsection{User-friendly}

\section{What is iCampo/LingSync?}

\subsection{Current functions -GINA }
\subsubsection{Data entry and import}
\subsubsection{Auto-glosser}
\subsubsection{Search}
\subsubsection{Sharing corpora, activity feed}
\subsubsection{Custom settings}
\subsubsection{Export}

\subsection{Desired future modules -GINA }
\subsubsection{Wiki-dictionary format export}
\subsubsection{Language learning module for android}
\subsubsection{Integration with prosodylab aligner}

\subsection{Collaborators - GRETCHEN} 

\section{How is iCampo/LingSync used so far?} 

\subsection{McGill-Listuguj partnership}

\subsection{Field methods classes} 

\subsection{Future users} 

\section{Conclusion - YULIYA} 

%\label{sec:intro}




\begin{exe}
	\ex The postman was rung by the door bell.
\end{exe}

\subsection{Subsection}

Lorem ipsum dolor sit amet, lectus faucibus, vitae tortor velit ut dolor integer, nunc sapiente mi, mauris nam magna, gravida et eu quisque. Etiam nibh mi tincidunt, libero euismod sed consectetuer amet, nam eget donec scelerisque sem sodales, curabitur et ut imperdiet libero leo pulvinar. Eros ipsum consectetuer donec imperdiet, dictum aliquet ac sollicitudin dui. Amet magna. Nunc nam. Pellentesque aut vestibulum, neque neque lacus quisque, curabitur bibendum fringilla pede magna non sociis, tempus gravida orci neque dictumst donec, etiam sit. \citep{MacFarlane:2005}

\subsubsection{Subsubsection}

Lorem ipsum dolor sit amet, lectus faucibus, vitae tortor velit ut dolor integer, nunc sapiente mi, mauris nam magna, gravida et eu quisque. Etiam nibh mi tincidunt, libero euismod sed consectetuer amet, nam eget donec scelerisque sem sodales, curabitur et ut imperdiet libero leo pulvinar. Eros ipsum consectetuer donec imperdiet, dictum aliquet ac sollicitudin dui. Amet magna. Nunc nam. Pellentesque aut vestibulum, neque neque lacus quisque, curabitur bibendum fringilla pede magna non sociis, tempus gravida orci neque dictumst donec, etiam sit.

\section{Introduction}
%\label{sec:intro}

Lorem ipsum dolor sit amet, lectus faucibus, vitae tortor velit ut dolor integer, nunc sapiente mi, mauris nam magna, gravida et eu quisque. Etiam nibh mi tincidunt, libero euismod sed consectetuer amet, nam eget donec scelerisque sem sodales, curabitur et ut imperdiet libero leo pulvinar. Eros ipsum consectetuer donec imperdiet, dictum aliquet ac sollicitudin dui. Amet magna. Nunc nam. Pellentesque aut vestibulum, neque neque lacus quisque, curabitur bibendum fringilla pede magna non sociis, tempus gravida orci neque dictumst donec, etiam sit.

\begin{exe}
	\ex The postman was rung by the door bell.
\end{exe}

\subsection{Subsection}

Lorem ipsum dolor sit amet, lectus faucibus, vitae tortor velit ut dolor integer, nunc sapiente mi, mauris nam magna, gravida et eu quisque. Etiam nibh mi tincidunt, libero euismod sed consectetuer amet, nam eget donec scelerisque sem sodales, curabitur et ut imperdiet libero leo pulvinar. Eros ipsum consectetuer donec imperdiet, dictum aliquet ac sollicitudin dui. Amet magna. Nunc nam. Pellentesque aut vestibulum, neque neque lacus quisque, curabitur bibendum fringilla pede magna non sociis, tempus gravida orci neque dictumst donec, etiam sit. \citep{MacFarlane:2005}

\subsubsection{Subsubsection}

Lorem ipsum dolor sit amet, lectus faucibus, vitae tortor velit ut dolor integer, nunc sapiente mi, mauris nam magna, gravida et eu quisque. Etiam nibh mi tincidunt, libero euismod sed consectetuer amet, nam eget donec scelerisque sem sodales, curabitur et ut imperdiet libero leo pulvinar. Eros ipsum consectetuer donec imperdiet, dictum aliquet ac sollicitudin dui. Amet magna. Nunc nam. Pellentesque aut vestibulum, neque neque lacus quisque, curabitur bibendum fringilla pede magna non sociis, tempus gravida orci neque dictumst donec, etiam sit.

\section{Introduction}
%\label{sec:intro}

Lorem ipsum dolor sit amet, lectus faucibus, vitae tortor velit ut dolor integer, nunc sapiente mi, mauris nam magna, gravida et eu quisque. Etiam nibh mi tincidunt, libero euismod sed consectetuer amet, nam eget donec scelerisque sem sodales, curabitur et ut imperdiet libero leo pulvinar. Eros ipsum consectetuer donec imperdiet, dictum aliquet ac sollicitudin dui. Amet magna. Nunc nam. Pellentesque aut vestibulum, neque neque lacus quisque, curabitur bibendum fringilla pede magna non sociis, tempus gravida orci neque dictumst donec, etiam sit.

\begin{exe}
	\ex The postman was rung by the door bell.
\end{exe}

\subsection{Subsection}

Lorem ipsum dolor sit amet, lectus faucibus, vitae tortor velit ut dolor integer, nunc sapiente mi, mauris nam magna, gravida et eu quisque. Etiam nibh mi tincidunt, libero euismod sed consectetuer amet, nam eget donec scelerisque sem sodales, curabitur et ut imperdiet libero leo pulvinar. Eros ipsum consectetuer donec imperdiet, dictum aliquet ac sollicitudin dui. Amet magna. Nunc nam. Pellentesque aut vestibulum, neque neque lacus quisque, curabitur bibendum fringilla pede magna non sociis, tempus gravida orci neque dictumst donec, etiam sit. \citep{MacFarlane:2005}

\subsubsection{Subsubsection}

Lorem ipsum dolor sit amet, lectus faucibus, vitae tortor velit ut dolor integer, nunc sapiente mi, mauris nam magna, gravida et eu quisque. Etiam nibh mi tincidunt, libero euismod sed consectetuer amet, nam eget donec scelerisque sem sodales, curabitur et ut imperdiet libero leo pulvinar. Eros ipsum consectetuer donec imperdiet, dictum aliquet ac sollicitudin dui. Amet magna. Nunc nam. Pellentesque aut vestibulum, neque neque lacus quisque, curabitur bibendum fringilla pede magna non sociis, tempus gravida orci neque dictumst donec, etiam sit.

\section{Introduction}
%\label{sec:intro}

Lorem ipsum dolor sit amet, lectus faucibus, vitae tortor velit ut dolor integer, nunc sapiente mi, mauris nam magna, gravida et eu quisque. Etiam nibh mi tincidunt, libero euismod sed consectetuer amet, nam eget donec scelerisque sem sodales, curabitur et ut imperdiet libero leo pulvinar. Eros ipsum consectetuer donec imperdiet, dictum aliquet ac sollicitudin dui. Amet magna. Nunc nam. Pellentesque aut vestibulum, neque neque lacus quisque, curabitur bibendum fringilla pede magna non sociis, tempus gravida orci neque dictumst donec, etiam sit.

\begin{exe}
	\ex The postman was rung by the door bell.
\end{exe}

\subsection{Subsection}

Lorem ipsum dolor sit amet, lectus faucibus, vitae tortor velit ut dolor integer, nunc sapiente mi, mauris nam magna, gravida et eu quisque. Etiam nibh mi tincidunt, libero euismod sed consectetuer amet, nam eget donec scelerisque sem sodales, curabitur et ut imperdiet libero leo pulvinar. Eros ipsum consectetuer donec imperdiet, dictum aliquet ac sollicitudin dui. Amet magna. Nunc nam. Pellentesque aut vestibulum, neque neque lacus quisque, curabitur bibendum fringilla pede magna non sociis, tempus gravida orci neque dictumst donec, etiam sit. \citep{MacFarlane:2005}

\subsubsection{Subsubsection}

Lorem ipsum dolor sit amet, lectus faucibus, vitae tortor velit ut dolor integer, nunc sapiente mi, mauris nam magna, gravida et eu quisque. Etiam nibh mi tincidunt, libero euismod sed consectetuer amet, nam eget donec scelerisque sem sodales, curabitur et ut imperdiet libero leo pulvinar. Eros ipsum consectetuer donec imperdiet, dictum aliquet ac sollicitudin dui. Amet magna. Nunc nam. Pellentesque aut vestibulum, neque neque lacus quisque, curabitur bibendum fringilla pede magna non sociis, tempus gravida orci neque dictumst donec, etiam sit.

\section{Introduction}
%\label{sec:intro}

Lorem ipsum dolor sit amet, lectus faucibus, vitae tortor velit ut dolor integer, nunc sapiente mi, mauris nam magna, gravida et eu quisque. Etiam nibh mi tincidunt, libero euismod sed consectetuer amet, nam eget donec scelerisque sem sodales, curabitur et ut imperdiet libero leo pulvinar. Eros ipsum consectetuer donec imperdiet, dictum aliquet ac sollicitudin dui. Amet magna. Nunc nam. Pellentesque aut vestibulum, neque neque lacus quisque, curabitur bibendum fringilla pede magna non sociis, tempus gravida orci neque dictumst donec, etiam sit.

\begin{exe}
	\ex The postman was rung by the door bell.
\end{exe}

\subsection{Subsection}

Lorem ipsum dolor sit amet, lectus faucibus, vitae tortor velit ut dolor integer, nunc sapiente mi, mauris nam magna, gravida et eu quisque. Etiam nibh mi tincidunt, libero euismod sed consectetuer amet, nam eget donec scelerisque sem sodales, curabitur et ut imperdiet libero leo pulvinar. Eros ipsum consectetuer donec imperdiet, dictum aliquet ac sollicitudin dui. Amet magna. Nunc nam. Pellentesque aut vestibulum, neque neque lacus quisque, curabitur bibendum fringilla pede magna non sociis, tempus gravida orci neque dictumst donec, etiam sit. \citep{MacFarlane:2005}

\subsubsection{Subsubsection}

Lorem ipsum dolor sit amet, lectus faucibus, vitae tortor velit ut dolor integer, nunc sapiente mi, mauris nam magna, gravida et eu quisque. Etiam nibh mi tincidunt, libero euismod sed consectetuer amet, nam eget donec scelerisque sem sodales, curabitur et ut imperdiet libero leo pulvinar. Eros ipsum consectetuer donec imperdiet, dictum aliquet ac sollicitudin dui. Amet magna. Nunc nam. Pellentesque aut vestibulum, neque neque lacus quisque, curabitur bibendum fringilla pede magna non sociis, tempus gravida orci neque dictumst donec, etiam sit.

\custombib{\bibliography{caml2012}}

\end{document}
