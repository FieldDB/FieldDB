\documentclass[12 pt]{article}
\usepackage{fullpage}
\usepackage{rotating}
\usepackage{tipa}
\usepackage{amsmath}
\usepackage{linguex}
\usepackage{enumerate}
\usepackage[sc]{mathpazo}
\linespread{1.05}         % Palatino needs more leading (space between lines)
\usepackage[T1]{fontenc}

\hyphenpenalty= 7000 

\title{Appendix \\  FieldDB programming resources}
\author{}
\date{}

\begin{document}
\maketitle{} 



\section {List of known programmers}

The appendix contains short bios of  programmers who have worked on the FieldDB project.

\subsection{Theresa Deering M.Sc.}

Theresa Deering has a  Bachelor's in Computer Science from Malaspina and a Master's in Computer Science from McGill University. She began her academic career in Mathematics and has since won the Governor General's Academic Medal twice, as well as held numerous NSERC scholarships throughout her academic career. Her programming experience ranges from obscure machine learning algorithms to interactive web development.  She is an active OpenSource contributor to BigData database implementations for offline WebApp databases. Her thesis focused on the Least-Used Direction pivot rule for the Simplex Method of solving linear programs.


\subsubsection{Languages - Human \& Machine}
\begin{description}
\item [
Human Languages] Native: English (British Columbia). Immersion: French (Montr\'eal), Japanese (Tokyo, Kyoto)
\item [
Machine Languages] -rwx: Java, Flex/ActionScript, ASP.NET, C\#, C, C++, Perl, Bash, LaTEX, Sybase, Oracle, XML, HTML -r-x: PHP, Git, R, PostgreSQL, Scheme
\end{description}

\subsubsection{Publications/Presentations}
\begin{itemize}
\item "The Least-Used Direction Pivot Rule on Acyclic Unique Sink Orientations." Master's Thesis. McGill University, Montr\'eal, Qu\'ebec. July 2010.
\item "Worst-case Behaviour of History Based Pivot Rules on Acyclic Unique Sink Orientations of Hypercubes." Y. Aoshima, D. Avis, T. Deering, Y. Matsumoto, S. Moriyama. Submitted to AAAC. October 2011.
\end{itemize}


\subsection{Gina Cook M.A.}

Gina Cook received her Masters in Fieldlinguistics from the University of Delaware \& DESS in Computer Science and Software Engineering from Concordia Univeristy. She has worked as a computational linguist for companies such as Nuance and Idilia.  She is an active contributor to OpenSource projects on GitHub which focus on integrating existing OpenSource libraries for Speech Recognition, Natural Language Processing, Eye Gaze analysis and Acoustic analysis into Android tablet applications, and into the data entry process for field linguists.



\subsubsection{Languages - Human \& Machine}
\begin{description}
\item [
Human Languages] Fieldwork: Hindi/Urdu (Lahore), Punjabi (Lahore), Korean (Ulsan), Mongolian (Hohhot), Turkish (Istanbul), Tunisian Arabic (Tunis), Romanian (Bucharest), Malay (Kuala Lumpur, Kuching), Czech (Brno), Pasto (Kabul), Inuktitut (Iqaluit) Coursework: French, Spanish, German, Russian, Ancient Greek, Tok Pisin, Sanskrit
\item [
Machine Languages] -rwx: LATEX (tikz), Java (Android), Javascript (Node.js), Git, MySQL, HTML5 (Offline Apps) -r-x: Praat (Visualization), Bash (Automation), XML (Ontologies), R, Groovy (NLP), SVN, Python (Scraping), Perl (RegEx), PHP (Web), C++ (Image Processing)

\end{description}

\subsubsection{Publications/Presentations}
\begin{itemize}

\item "Aphasia Assessment on Android: recording voice, eye-gaze and touch for the BAT." (with A. Marquis \& A. Achim). Poster at Academy of Aphasia 49th Annual Meeting, Mont\'eal, Qu\'ebec. October 2011.
\item "Eliciting evolving information structure and audienceless vs. audience oriented prosodies: experimentation on Android tablets." (with S. Kattoju). Poster at ETAP2 � Experimental and Theoretical Advances in Prosody, Montr\'eal, Qu\'ebec. September 2011.
\item "PDFtoAudioBook Android app" (Java, XML).Canadian University Software Engineering Conference (CUSEC) DemoCamp, Montr\'eal, Qu\'ebec. January 2011.
\item "Word features and word concatenation."Sixth Interdisciplinary Graduate Student Research Symposium, McGill University, Montr\'eal, Qu\'ebec. March 2009.
\item "The Structure of Long Distance Agreement in Hindi/Urdu." Invited Lecture in Advanced Syntax, Concordia University, Montr\'eal, Qu\'ebec. November 2007.
\item "The Phonological/Phonetic status of Productive Palatalization in Romanian." (with L. Spinu). Presented at the Seoul International Conference on Linguistics, Seoul National University, Seoul, South Korea. July 2006.
\end{itemize}



\subsection{M.E. Cathcart Ph.D.}

M.E. Cathcart completed her PhD at the University of Delaware with a dissertation grant funded by the National Science Foundation (NSF) for her fieldwork in Cusco, Peru on Quechua. In addition, she also has a background of coursework in computational linguistics, at the University of Delaware and at the Linguistic Society of America's Summer Institute. She is an active OpenSource contributor for projects to gamify/crowd source data collection, both in fieldwork and psycholinguistic experiments.


\subsubsection{Languages - Human \& Machine}
\begin{description}
\item [
Human Languages] Native: Spanish (Sonora, Mex) English (California) Fieldwork: Quechua (Cusco), Finnish (Helsinki), Albanian (Tosk), Bulgarian (Sofia), Malay (Kuala Lumpur), and Serbo-Croatian (Sarajevo, Belgrade) Cousework: French, Italian, and Quechua
\item [
Machine Languages] -rwx: LATEX, Git, HTML5, R -r-x: Java, Javascript, Praat, Perl

\end{description}


\subsubsection{PublicationsPresentations}

\begin{itemize}
\item "Affected Arguments Cross-linguistically." S. Bosse, B. Bruening, M.E. Cathcart, A. E. Peng, M. Yamada. In: Tadic, M. Dimitrova-Vulchanova, M., Koeva, S. (eds.): FASSBL 6 The Sixth International Conference on Formal Approaches to South Slavic and Balkan Languages. 2008 (Proceedings) pp. 41-47.
\item "A New Grammatical Category: Impulsatives." Penn Linguistics Colloquium, Philadelphia. March 2010
%Eliciting data for dissertation on Impulsatives: functional morpheme in context
\item"The Syntax and Semantics of Desideratives in Albanian." Georgetown Linguistics Society, Washington, D.C. February 2010
\item "Bi-Eventivity \& Affecting Arguments." S. Bosse, B. Bruening, M.E. Cathcart, H.-j. Cheng, A. E. Peng, M. Yamada. Formal Approaches to South Slavic and Balkan Languages, Dubrovnik, Croatia. September 2008.
\item "Bi-Eventive Affect." S. Bosse, B. Bruening, M.E. Cathcart, H.-j. Cheng, A. E. Peng, M. Yamada.TEAL, Potsdam, Germany. September 2008.
\end{itemize}




\end{document}  



