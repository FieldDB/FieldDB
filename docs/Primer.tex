\documentclass[12 pt]{article}
\usepackage{fullpage}
\usepackage{tipa}
\usepackage{amsmath}
\usepackage{linguex}
\usepackage{enumerate}
\usepackage[sc]{mathpazo}
\linespread{1.05}         % Palatino needs more leading (space between lines)
\usepackage[T1]{fontenc}


\title{The Field Linguists' App}
\author{}
\date{}

\begin{document}
\maketitle{}

\section {Project Abstract}

The FieldLinguists' App aims to provide language researchers with an OpenSource database where they can securely enter, store, and organize linguistic data, customized to their particular needs. The application runs on three different systems (Mac, Linux, and Windows) and is accessible everywhere as it is suitable for both online and offline use. Furthermore, the project was created with collaborative goals in mind; the application will make data syncable and sharable with other researchers in the database, who can be assigned to teams and have discussions via comments. The system will also have a simple and friendly user interface, allowing participants to drag and drop files with the datum (audio, video, text), or record it directly into the database when using the Android app. Most importantly, the application is designed intuitively and theory free, so it is not necessary to be a field linguist or programmer to figure out how it works.

\section {Statement of Need}

The statement of need should describe the problem that the project will attempt to address. Also, 
describe the population that will be served.  

\section {App Description}

Describe the project or program and provide information on how it will be implemented. Include 
information on what will be accomplished and the desired outcome. 

\section {Goals \& Objectives}

Describe the project objectives in measurable terms that address the academic and technological needs of language teachers, linguists, etc. (it's for linguists but other people can benefit from it as well i.e. help create dictionaries for endangered languages, which will benefit the communities of these languages; L2 acq. teachers).  

\section {Timeline}

How long does it take us to make it? Project progress over time. What we've done and what we'll do.

\section {Budget}

Include in the budget all expenses for the app, including necessary training costs. Mention any co-funding that you are using from other sources. You may want to include a brief narrative of expenses along with a table of individual cost components. 

\section {Evaluation}

Provide information on the metrics that will be used to determine the effectiveness of the project. It has to meet certain data management criteria -- any scientist can discover and use the data over time; it will benefit the scientific community. We are abiding by the best practice standards set by DataOne.   

\section {Staff \& Organizational Information}

Include the staff qualifications, certifications, and skills. Describe the organization and include 
information indicating the organization�s capacity to implement and sustain the program--i.e. info about iLanguage Lab LTD. CVs from people from the company, experience, etc.

Example format: 

\exg. n-tuop'ti-m\\
1-window-POSS.SG\\
`My window'


\end{document}