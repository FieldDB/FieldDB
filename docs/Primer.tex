
\documentclass[12 pt]{article}
\usepackage{fullpage}
\usepackage{rotating}
\usepackage{tipa}
\usepackage{amsmath}
\usepackage{linguex}
\usepackage{enumerate}
\usepackage[sc]{mathpazo}
\linespread{1.05}         % Palatino needs more leading (space between lines)
\usepackage[T1]{fontenc}


\title{The FieldLinguists' App}
\author{}
\date{}

\begin{document}
\maketitle{}

\section {Project Abstract}

The FieldLinguists' App is an OpenSource database that will allow language researchers to securely enter, store, organize, annotate, and share linguistic data. The application will be accessible everywhere; it will run on three different systems (Mac, Linux, and Windows)  and will be suitable for both online and offline use.  Furthermore, the application will be created with collaborative goals in mind;  data will be syncable and sharable with other researchers.  Researchers can form teams that contribute to a single corpus, where team members can modify and discuss the data from within the application. The system will also have a simple and friendly user interface, allowing participants to drag and drop files (audio, video, text), or record them directly into the database when using the Android app. In addition, the application will have import and export capabilities for multiple file types.  Most importantly, the application is designed intuitively and theory free, so it is not necessary to be a field linguist or programmer to figure out how it works.

\section {Statement of Need}

The FieldLinguists' App is conceived out of the needs of language researchers doing fieldwork. Linguistic fieldwork often requires researchers to travel to places where a stable connection to the internet is not guaranteed. Also, it often involves a group of researchers contributing to building a single database.
 An ideal linguistic database should therefore work both on- and off-line as well as make it easy to share and integrate data. 
 
There are existing programmes/software used for linguistic fieldwork, 
however, they fall short in providing features necessary for collaborative fieldwork. 
For example, some web-based databases (e.g.  {\it Karuk Dictionary and Texts} http://linguistics.berkeley.edu/{\textasciitilde}ka
ruk/links.php,   {\it The Washo Project} http://washo.uchicago.edu/dictionary/dictionary.php) work only online, hence it is impossible for researchers to enter new data or search the database while in the field where the internet is unavailable.  

Non web-based software such as {\it Toolbox} (http://www.sil.org/computing/toolbox/) and {\it FLEx/FieldWorks} (http://fieldworks.sil.org/flex/)
are excellent in terms of annotating data and building data into various types (corpus, grammar or lexicon). Nonetheless, integration of data taken by individual researchers is not well-automated and takes considerable work by researchers. Moreover, they have a particular preference of platforms (either PC or Linux, but not both), which causes further difficulty in data sharing among researchers using different platforms.  

General purpose database software such as FileMaker Pro can be customized for the purpose of language research. However, it demands researchers to learn the software, and research teams often need to hire a programmer to customize the software for their research purposes. The stand-alone nature of such software makes data sharing and integration difficult as well. 

The existing linguistic database programmes, although useful, have various shortfalls that would hinder collection and integration of data. Some of them are constrained by the internet accessibility or by the computer platform types. Some others demand extra human work in order to integrate data. A database application whose functionality is not limited by such constraints is much anticipated by language researchers. 


\section {App Description}

The FieldLinguists' App will enable those interested in language research, preservation, and documentation to securely enter, store, organize, annotate, and share linguistic data. Moreover, the application will be easily customizable to fit specific needs. To accomplish these tasks, the database will be equipped with a variety of features. 
The following functional requirements are based on a few important considerations where most existing fieldlinguistics/corpus linguistics databases applications fall short. 


\subsection{Functionality}

This application will be able to perform the necessary functions needed by fieldlinguists. The dashboard will be composed of several widgets. The Data Entry widget will be the primary focus, containing three core fields customary for a gloss format (utterance, gloss, translation). In addition to these, researchers will be able to add customized fields, such as phonetic transcription or context for an utterance.  Researchers can even upload audio files and link them to the appropriate data. 

Furthermore, researchers can add tags for categorization and mark the status of each individual data entry as "Checked" or "To be checked with consultant", which further aids organization and reduces the number of errors that inevitably occur during research. Other functions such as importing data and exporting data into various formats aid efficiency and convenience. 

Another widget will be an Activity Feed View displaying the most recent changes. This widget allows researchers to keep up to date on their teams' activity.  The activity feed will display items such as recent additions to the corpus, comments made on data entries, and recent edits. 

Finally, the application will have a powerful search function that will expand into a data list that is contained in its own widget. Data lists can be sorted, saved and can be used for batch operations such as exporting or converting into LaTeX.




\begin{itemize}


\item {\bf Modern}

\begin{itemize}
\item { \bf Simple} The system will be designed to replace Word Documents or LaTeX documents which is a very common way fieldlinguists store data because it requires no training, doesn't require a complicated set-up for data categories, and takes no time to add new categories.  

\item {\bf Attractive} The system will have a modern design like many of the popular websites such as Google and be customizable so that the user can include a picture of where they are doing research as a background.

\end{itemize}

\item {\bf Powerful}
\begin{itemize}
\item {\bf Smart.} The application will guess what users do most often, and automate the process for them. Most importantly, the system will have automated predictable glossing information.
\item {\bf  Searchable.} The application will be designed for search as this is one of the most fundamental tasks a language researcher must be able to do. The search will go far beyond traditional string matches and database indexes; it will be able to display data in context.
\end{itemize}

\item{\bf Data-Centric}
\begin{itemize}
\item {\bf Atheoretical.} The application will not include categories or linguistic frameworks or theoretical constructs that must be tied to the data.  The application will allow an analysis to develop organically as data collection proceeds, as opposed to imposing a particular construct upon entry.  Researchers will be able to set and change their own categories for the data whenever they choose to.
\item {\bf Collaborative.} The system will have users and teams, and permissions for corpora. Permissions will ensure that data can be safely shared and edited by multiple users. Moreover, the corpus will be versioned so that users can track changes and revert mistakes.
\item {\bf Sharable.} The application will allow researchers to share their data with anyone interested in their work.
\end{itemize}


\item{\bf Accessible}
\begin{itemize}
\item { \bf Cross-Platform.} The application will run on Mac, Linux, and Windows computers. The application will be installable as a Chrome extension and available on any device that runs a browser.
\item  {\bf Portable.} Touch tablets are one of the easiest tools to carry and use in field; they have a long battery life; they can play videos or show images for the consultant to elicit complicated contexts; and they permit recording audio and video and direct publishing to YouTube and/or other services. Furthermore, Android tablets are particularly easy to program and integrate the microphone directly into the database.
\item {\bf Work offine.} Running a webapp offline will have considerable consequences for how data is stored, how data is retrieved, and how much data can be used while offline. Most browsers have limits on the amount of data a webapp can store offline. By delivering a version of the app in a Chrome extension, which has permission to have unlimited storage, researchers will be able to have a significant portion of their data at their fingertips, regardless of the location.
\end{itemize}



\item{\bf Open}
\begin{itemize}
\item {\bf OpenData}. Corpora often contain sensitive information, informant stories and other information which must be kept confidential. Having confidential data in plain text in a corpus forces the entire corpus to be kept confidential. Instead, the system will encrypt confidential data and store the data in the corpus encrypted. To access the plain text the user will have to log in and use a password to decrypt the data. This design has important ramifications for exporting data, and for editing the data outside the application.
\item { \bf OpenSource}. Being OpenSource allows departments to install and customize the database application to tailor specific needs without worry that the company behind the software will disappear or stop maintaining the software. In addition, OpenSourcing the software on GitHub will allow linguists with scripting or programming experience to contribute back to the software to make it more customized to their needs, language typologies, or linguistics research areas.
\item {\bf Unicode}. Encoding problems and losing data should be behind us in the days of unicode. However, many existing fieldlinguistics databases were built in programming languages that did not support unicode, so the unicode support is dangerously fragile.
\end{itemize}

\end{itemize}


\section {Goals \& Objectives}

The principal goal of The FieldLinguists' App is to help language researchers collect and organize linguistic data and to facilitate collaborative research work. The main objectives are to provide: 

\begin{itemize} 
\item A self-explanatory, easy-to-use user interface so that researchers can start using the application at the time the installation is completed.  
\item Both on- and off-line functionality so that the fieldwork is not constrained by the internet accessibility. 
\item Customizable data entry fields to accommodate particular requirements of a research.  
\item Data sharing and integration functions to facilitate collaboration among researchers and between researchers and language consultants. 

\end{itemize} 


Although it is designed primarily for linguists, the application will equally be useful for researchers documenting endangered languages and/or creating dictionaries/grammar books for minority languages, as well as language teachers creating educational materials.  



\section {Staff \& Organizational Information}

 iLanguage Lab is a Montreal based company that develops tools in the form of experimentation and data collection apps for Android and Chrome in collaboration with researchers at UdeM, UQAM, McGill and Concordia. Previous research applications includes the Bilingual Aphasia Test, AuBlog, OPrime and SpyOrNot.  Furthermore, iLanguage Lab has a background in assisting researchers obtain results and publications.  The AuBlog application was employed to investigate evolving information structure and audienceless vs. audience oriented prosodies and culminated in a poster presented at Experimental and Theoretical Advances in Prosody Conference. The Bilingual Aphasia Test led to a presentation at the  Academy of Aphasia 49th Annual Meeting on  Aphasia Assessment on Android: recording voice, eye-gaze and touch for the BAT and a publication in the Academy of Aphasia. 

\subsection{Gina Cook M.A.}

Gina Cook received her Masters in Field Linguistics \& DESS in Computer Science and has worked as a computational linguist for companies such as Nuance and Idelia.  She founded iLanguage Lab with a vision to develop computational tools to help researchers as opposed to consumers.  She is an active contributor to OpenSource projects on GitHub focusing on integrating existing OpenSource libraries for Speech Recognition, Natural Language Processing, Eye Gaze analysis and Acoustic analysis into Android tablet applications.

\subsubsection{Publications}
\begin{itemize}

\item "Aphasia Assessment on Android: recording voice, eye-gaze and touch for the BAT." (with A. Marquis \& A. Achim). Poster at Academy of Aphasia 49th Annual Meeting, Mont\'eal, Qu\'ebec. October 2011.
\item "Eliciting evolving information structure and audienceless vs. audience oriented prosodies: experimentation on Android tablets." (with S. Kattoju). Poster at ETAP2 � Experimental and Theoretical Advances in Prosody, Montr\'eal, Qu\'ebec. September 2011.
\item "PDFtoAudioBook Android app" (Java, XML).Canadian University Software Engineering Conference (CUSEC) DemoCamp, Montr\'eal, Qu\'ebec. January 2011.
\item "Word features and word concatenation."Sixth Interdisciplinary Graduate Student Research Symposium, McGill University, Montr\'eal, Qu\'ebec. March 2009.
\item "The Structure of Long Distance Agreement in Hindi/Urdu." Invited Lecture in Advanced Syntax, Concordia University, Montr\'eal, Qu\'ebec. November 2007.
\item "The Phonological/Phonetic status of Productive Palatalization in Romanian." (with L. Spinu). Presented at the Seoul International Conference on Linguistics, Seoul National University, Seoul, South Korea. July 2006.
\end{itemize}


\subsection{M.E. Cathcart Ph.D.}

M.E. Cathcart completed her PhD at the University of Delaware with a dissertation grant funded by the National Science Foundation (NSF) for her fieldwork in Cusco, Peru on Quechua. In addition, she also has a background of coursework in computational linguistics, at the University of Delaware and at the Linguistic Society of America�s Summer Institute.

\subsubsection{Publications}

\begin{itemize}
\item "Affected Arguments Cross-linguistically." S. Bosse, B. Bruening, M.E. Cathcart, A. E. Peng, M. Yamada. In: Tadic, M. Dimitrova-Vulchanova, M., Koeva, S. (eds.): FASSBL 6 The Sixth International Conference on Formal Approaches to South Slavic and Balkan Languages. 2008 (Proceedings) pp. 41-47.
\item "A New Grammatical Category: Impulsatives." Penn Linguistics Colloquium, Philadelphia March 2010
?Eliciting data for dissertation on Impulsatives: functional morpheme in context
\item"The Syntax and Semantics of Desideratives in Albanian." Georgetown Linguistics Society, Washington, D.C. February 2010
\item "Bi-Eventivity \& Affecting Arguments." S. Bosse, B. Bruening, M.E. Cathcart, H.-j. Cheng, A. E. Peng, M. Yamada. Formal Approaches to South Slavic and Balkan Languages, Dubrovnik (Croatia), 25-28 September 2008.
\item "Bi-Eventive Affect." S. Bosse, B. Bruening, M.E. Cathcart, H.-j. Cheng, A. E. Peng, M. Yamada.TEAL, Potsdam (Germany), 10-11 September 2008.
\end{itemize}


\subsection{Theresa Deering M.A.}

Theresa Deering has a  Bachelor's in Computer Science from Malaspina and a Master's in Computer Science from McGill University. Her thesis focused on the Least-Used Direction pivot rule for the Simplex Method of solving linear programs.

\begin{itemize}
\item "The Least-Used Direction Pivot Rule on Acyclic Unique Sink Orientations." Master's Thesis. McGill University, Montr\'eal, Qu\'ebec. July 2010.
\item "Worst-case Behaviour of History Based Pivot Rules on Acyclic Unique Sink Orientations of Hypercubes." Y. Aoshima, D. Avis, T. Deering, Y. Matsumoto, S. Moriyama. Submitted to AAAC. October 2011.
\end{itemize}

\section {Budget \& Timeline}

The FieldLinguists' App is composed of eight modules and thus the cost is divided into eight major components. In addition, a separate price is given for software architecture and for 1 year of user support and project growth, which is needed to make a longterm viable and useful tool that fieldlinguists can adopt for their labs or for their field methods courses. The cost is calculated by determining the time in hours and multiplying by \$42-58, the average rate of a software developer in Montr\'eal. 

Since the budget is dependent on how long it takes to complete the modules, it is possible to focus on each module separately, thus reducing the time of completion. 

 


\begin{table}[htbp]
\begin{center}
  \begin{tabular}{ | lcl | }
\hline
    Module & Weeks & Price\\ 
\hline
    Software Architecture & 0.5  & \$1,555.20  \\ 
    Collaboration Module & 2.5  & \$5,728.32  \\ 
    Corpus Module & 4.2 & \$9,201.60 \\
     Web Spider Module  & 2.5 & \$2,177.28\\
      Lexicon Module & 2.0 & \$7,340.54 \\ 
     Phonological Search Module & 3.0 & \$2,177.28 \\
    Dictionary Module & 22.3 & \$18,781.63 \\
        Glosser Module & 19.7 & \$17,770.75 \\
Aligner Module & 10.2 & \$ 9,787.39\\ 
User Support & 21.1 & \$30,246.70 \\
TVS and TPQ &  & \$10,833.32 \\
Total &  88 &\$83,176.04\\

\hline
  \end{tabular}
  \caption{Project Summary}
  \label{tab:label}
  \end{center}
\end{table}



\newpage
\subsection{Collaboration Module}

\footnotesize
\begin{table}[htbp]
\begin{center}
  \begin{tabular}{ | lcl | }
\hline
Iteration&	Hours&	Technology	\\
\hline
Software Architecture Design&	20&	Software Engineering	\\ 
Collaboration API on central server&	30&	Software Engineering\\ 
Users Model&	15&	Javascript	\\ 
Informant Model&	15&	Javascript	\\ 
Team Model&	15&	Javascript	\\ 
Bot Model&	15&	Javascript	\\ 
User Activity Model&	8&	Javascript	\\ 
Team Feed Widget&	25&	HTML5	\\ 
User list item Widget&	16&	HTML5	\\ 
Team Preferences Widget&	8&	HTML5	\\ 
User Profile Widget&	8&	HTML5	\\ 
User Tests&	30&	Javascript \\ 
Informant Tests&	30&	Javascript	\\ 
Team Tests&	30&	Javascript	\\ 
Android Deployment&	15&	Java	\\ 
Chrome Extension Deployment&	20&	Javascript \\ 
Heroku Deployment&	5&	Integration	\\ 
%\# of weeks with 3 full time personnel&	2.5416666667& \\ 
\hline
  \end{tabular}
  \caption{The Collaboration Module is used to permit collaboration with teams and users. }
  \label{tab:label}
  \end{center}
\end{table}





\subsection{Corpus Module}

\footnotesize
\begin{table}[htbp]
\begin{center}
  \begin{tabular}{ | lcl | }
\hline

Iteration&	Hours&	Technology	\\
\hline
Software Architecture Design&	20&	Software Engineering	\\ 
Corpus API on corpus server&	20&	Software Engineering\\ 
Corpus Model&	8&	Javascript	\\ 
Session Model&	8&	Javascript	\\ 
Datum Model&	8&	Javascript	\\ 
Datum status model&	8&	Javascript	\\ 
DataList Model&	8&	Javascript	\\ 
Confidential datum encrypter&	16&	Javascript	\\ 
Audio upload and play logic&	8&	Javascript	\\ 
Corpus DB  implementation on Android&	20&	Java	\\ 
Corpus DB  implementation on Chrome&	20&	Javascript	\\ 
Corpus DB  implementation on Node.js&	20&	Javascript \\ 
Corpus versioning Logic&	25&	Javascript	\\ 
Corpus Preferences Widget&	6&	HTML5	\\ 
Session Preferences Widget&	6&	HTML5	\\ 
Datum Preferences Widget&	20&	HTML5 \\ 
Datum Status Preferences Widget&	16&	HTML5	\\ 
DataList Preferences Widget&	6&  HTML5 \\ 
Corpus sync logic&	10&  Javascript \\ 
Corpus diff Widget (to show before sync)&	10&  HTML5 \\ 
Insert Unicode Character Widget&	10&  HTML5 \\ 
Corpus Details Widget&	6&  HTML5 \\
Session Details Widget&	6&  HTML5 \\  
Datum Details Widget&	20&  Javascript \\ 
DataList Widget&	30&  Javascript \\ 
Global Search logic&	30&  Javascript \\ 
Power Search logic&	80&  Javascript \\ 
Corpus Tests&	5&  Javascript \\ 
Session Tests&	10&  Javascript \\ 
Datum Tests&	10&  Javascript \\ 
Datum Status Tests&	10&  Javascript \\ 
DataList Tests&	20&  Javascript \\ 
Heroku Deployment&	5&  Integration \\ 
\hline
  \end{tabular}
 \caption{The Corpus Module is used to sync, share, edit, tag, categorize and open data. }
  \label{tab:label}
  \end{center}
\end{table}
%# of weeks with 3 full time personnel	4.20833333333333



\newpage
\subsection{Web Spider Module}

\footnotesize
\begin{table}[htbp]
\begin{center}
  \begin{tabular}{ | lcl | }
\hline
Iteration&	Hours&	Technology	\\
\hline
Corpus Visualization Widget&	40&	HTML5	\\ 
Web Spider Training Logic&	60&	Java	\\ 
\hline
  \end{tabular}
  \caption{This module is a subportion of the Corpus Module.}
  \label{tab:label}
  \end{center}
\end{table}
%# of weeks with 1 full time personnel	2.5


\newpage
\subsection{Lexicon Module}

\footnotesize
\begin{table}[htbp]
\begin{center}
  \begin{tabular}{ | lcl | }
\hline

Iteration&	Hours&	Technology	\\
\hline
Software Architecture Design&	20&	Software Engineering	\\ 
Lexicon API on Lexicon server&	20&	Software Engineering\\ 
Lexicon Model&	6&	Javascript	\\ 
Morpheme Model&	6&	Javascript	\\ 
Allomorph Model&	6&	Javascript	\\ 
Gloss Model&	6&	Javascript	\\ 
Orthography Model&	16&	Javascript	\\ 
Lexicon DB  implementation on Android&	20&	Java	\\ 
Lexicon DB  implementation on Chrome&	20&	Javascript	\\ 
Lexicon DB  implementation on Node.js&	20&	Javascript \\ 
Lexicon versioning Logic&	10&	Javascript	\\ 
Lexicon Preferences Widget&	6&	HTML5	\\ 
Morpheme Tests&	6&	Javascript	\\ 
Allomorph Tests&	6&	Javascript \\ 
Gloss Tests&	6&	Javascript	\\ 
Orthography Tests&	8&  Javascript \\ 
Lexicon Analysis Widget&	10&  HTML5 \\ 
Lexicon sync logic&	10&  Javascript \\ 
Lexicon diff Widget (to show before sync)&	10&  HTML5 \\ 
Lexicon Details Widget&	6&  HTML5 \\
Lexicon Tests&	12&  Javascript \\  
Heroku Deployment&	5&  Integration \\ 
\hline
  \end{tabular}
 \caption{The Lexicon Module is used to house, and read lexicon entries to be used for the glosser.}
  \label{tab:label}
  \end{center}
\end{table}
%# of weeks with 3 full time personnel	1.95833333333333

\newpage
\subsection{Phonological Search Module}

\footnotesize
\begin{table}[htbp]
\begin{center}
  \begin{tabular}{ | lcl | }
\hline
Iteration&	Hours&	Technology	\\
\hline
Phonology Ontology for phonological search&	60&	Java	\\ 
Lexicon Visualization Widget&	40&	Javascript	\\ 
Lexicon Editing Widget&	20&	Javascript	\\ 
\hline
  \end{tabular}
 \caption{This module is a subportion of the Lexicon Module.}
  \label{tab:label}
  \end{center}
\end{table}
%# of weeks with 1 full time personnel	3


\newpage
\subsection{Dictionary Module}

\footnotesize
\begin{table}[htbp]
\begin{center}
  \begin{tabular}{ | lcl | }
\hline

Iteration&	Hours&	Technology	\\
\hline
Software Architecture Design&	40&	Software Engineering	\\ 
Dictionary API on Lexicon server&	30&	Software Engineering\\ 
Semantic Model&	60&	Javascript	\\ 
Syntactic Model&	60&	Javascript	\\ 
Citation Model&	60&	Javascript	\\ 
Synonyms Model&	60&	Javascript	\\ 
Dictionary DB implementation&	80&	Integration	\\ 
Dictionary Training Logic&	80&	Java	\\ 
Web Spider Training Logic&	100&	Java	\\ 
Dictionary Preferences Widget&	8&	HTML5	\\ 
Dictionary WordNet Analysis Widget&	120&	HTML6	\\ 
Dialect Profile Widget&	8&	HTML5 \\ 
Semantic Tests&	30&	Javascript	\\ 
Syntactic Tests&	30&  Javascript \\ 
Citation Tests&	         30&  Javascript \\ 
Synonyms Tests&	30&  Javascript \\ 
Spider Tests&	30&	Java \\ 
Training Tests&	30&	Java	\\ 
Heroku Deployment&	5&  Integration \\ 
\hline
  \end{tabular}
 \caption{The Dictionary Module is used to share the lexicon in the form of a WordNet/Wiktionary dictionary with the language community as required by some grants.}
  \label{tab:label}
  \end{center}
\end{table}
%# of weeks with 1 full time personnel	22.275


\newpage
\subsection{Glosser Module}

\footnotesize
\begin{table}[htbp]
\begin{center}
  \begin{tabular}{ | lcl | }
\hline

Iteration&	Hours&	Technology	\\
\hline
Software Architecture Design&	40&	Software Engineering	\\ 
Glosser API on Lexicon server&	30&	Software Engineering\\ 
Morpheme Model&	15&	Javascript	\\ 
Allomorph Model&	15&	Javascript	\\ 
Gloss Model&	15&	Javascript	\\ 
Orthography Model&	30&	Javascript	\\ 
Glosser DB implementation&	80&	Integration	\\ 
Glosser Prediction Logic&	80&	Java	\\ 
Glosser Machine Learning Logic&	80&	Java	\\ 
Glosser Training Logic&	80&	Java	\\ 
Web Spider Training Logic&	80&	Java	\\ 
Glosser Preferences Widget&	8&	HTML5	\\ 
Morphological Analysis Widget&	40&	HTML6	\\ 
Dialect Profile Widget&	8&	HTML5 \\ 
Morpheme Tests&	30&	Javascript	\\ 
Allomorph Tests&	30&  Javascript \\ 
Gloss Tests&	         30&  Javascript \\ 
Orthography Tests&	30&  Javascript \\ 
Spider Tests&	30&	Java \\ 
Training Tests&	30&	Java	\\ 
Heroku Deployment&	5&  Integration \\ 
\hline
  \end{tabular}
   \caption{The Glosser Module is used to automatically gloss datum, smarter than the standard lexicon.}
  \label{tab:label}
  \end{center}
\end{table}
%# of weeks with 1 full time personnel	19.65


\newpage
\subsection{Aligner Module}

\footnotesize
\begin{table}[htbp]
\begin{center}
  \begin{tabular}{ | lcl | }
\hline

Iteration&	Hours&	Technology	\\
\hline
Software Architecture Design&	10&	Software Engineering	\\ 
Aligner API on Lexicon server&	10&	Software Engineering\\ 
Dictionary Model&	15&	Javascript	\\ 
Aligner DB implementation&	80&	Integration	\\ 
Aligner Machine Learning Integration&	80&	Java	\\ 
Aligner Preferences Widget&	8&	HTML5	\\ 
Audio Waveform Visualization logic&	30&	Javascript	\\ 
Audio Spectrogram Visualization logic&	?&	Javascript \\ 
Transcription User Interface&	80&	HTML5	\\ 
TextGrid export&	20&  Javascript \\ 
Dialect Profile Widget&	         8&  HTML5 \\ 
Orthography Tests&	30&  Javascript \\ 
Training Tests&	30&	Java	\\ 
Heroku Deployment&	5&  Integration \\ 
\hline
  \end{tabular}
  \caption{The Aligner Module is used to create TextGrids from the orthography and the audio files, used for prosody and phonetic analysis.}
  \label{tab:label}
  \end{center}
\end{table}
%# of weeks with 1 full time personnel	10.15

\newpage
\subsection{User Support}

\footnotesize
\begin{table}[htbp]
\begin{center}
  \begin{tabular}{ | lcl | }
\hline

Iteration&	Hours&	Technology	\\
\hline
Sample data&	30&	Linguistics	\\ 
Integrate software with sample data&	30&	Javascript\\ 
Screencasts on how to use the app(s)&	24&	Quicktime/YouTube	\\ 
Screencasts on how to modify the code&	40&	Quicktime/YouTube	\\ 
Server maintenance&	20&	Integration	\\ 
Monitor server costs and develop pricing plan&	100&   Business	\\ 
Answer user emails&	250&	Support	\\ 
Read twitter feeds and facebook channels&	100&	Support \\ 
Help IT/developers install and set up \\ the server on their department servers&	50&	Support	\\ 
Upgrade javascript/android libraries&	40&  Javascript \\ 
Amazon EC2 server CPU+Memory+Bandwidth&	         &  Server \\ 
Release new versions&	160&  Javascript \\ 
\hline
  \end{tabular}
 \caption{User Support includes 1 year of product support and project growth. It is needed to make a longterm viable and useful tool that field linguists can adopt for their labs or for their field methods courses.} 
  \label{tab:label}
  \end{center}
\end{table}
%# of weeks with 1 full time personnel	21.1














%%Include in the budget all expenses for the app, including necessary training costs. Mention any co-funding that you are using from other sources. You may want to include a brief narrative of expenses along with a table of individual cost components. 


\normalsize

\section {Evaluation}

The usefulness and effectiveness of the FieldLinguists' App will be evaluated against the following criteria described in DataONE best data management practices (http://www.data
one.org/sites/all/documents/DataONE\_BP\_Primer\_020212.pdf). 

\begin{itemize} 

\item Plan: The goal of the application is to help facilitate language researchers. It is crucial that the application does not impose any constraint on their research plans, including data management plans. 
\item Collect: Data are collected and organized with ease (self-explanatory UI, off-line functionality).   
\item Assure: The quality of the data is assured through checks and inspections (datum status, comment function). 
\item Describe: The application accommodates metadata categories that researchers choose to use (new data entry fields, tags). 
\item Preserve: Data are securely archived in the host server and confidential information (e.g. consultant's identity) is made inaccessible to the public.  
\item Discover: Data are searchable.  
\item Integrate: Data from disparate sources are combined into one consistent data set. Data are shareable (sync function, import/export functions). 
\item Analyze: The application organizes data in ways to help data analysis (data entry fields, tags).  

\end{itemize}





%
%Provide information on the metrics that will be used to determine the effectiveness of the project. It has to meet certain data management criteria -- any scientist can discover and use the data over time; it will benefit the scientific community. We are abiding by the best practice standards set by DataOne.   
%
%Example format: 
%
%\exg. n-tuop'ti-m\\
%1-window-POSS.SG\\
%`My window'
%
%
\end{document}

