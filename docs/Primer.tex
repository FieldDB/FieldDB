\documentclass[12 pt]{article}
\usepackage{fullpage}
\usepackage{tipa}
\usepackage{amsmath}
\usepackage{linguex}
\usepackage{enumerate}
\usepackage[sc]{mathpazo}
\linespread{1.05}         % Palatino needs more leading (space between lines)
\usepackage[T1]{fontenc}


\title{The Field Linguists' App}
\author{}
\date{}

\begin{document}
\maketitle{}

\section {Project Abstract}

Overview. 

\section {Statement of Need}

The statement of need should describe the problem that the project will attempt to address. Also, 
describe the population that will be served.  

This project is inspired by and developed from the needs of the linguists who are working on languages that 
are not extensively documented. Research on a less-studied language involves documentation of the language 
itself, which 

There are several existing programs/software used for the purpose of archiving linguistic data, 
some of them made by linguistic researchers themselves. Although such programmes/software have features 
useful for particular research purposes, there is not a single programme/software that comprehensively 
provides features necessary for collaborative field work.  

For example, existing web-based programmes (e.g. ) works only on-line, which makes it impossible for 
researchers to enter new data or search the database while they are in the field and have no access to the 
internet. 




{\it Karuk Dictionary and Texts} (http://linguistics.berkeley.edu/~karuk/links.php) 
{The Washo Project} (http://washo.uchicago.edu/dictionary/dictionary.php) 

They are web-based searchable database for Karuk and Washo languages respectively, made by the researchers. 
They work only online, which makes 

{\it Toolbox} (http://www.sil.org/computing/toolbox/)
{\it FLEx/FieldWorks} (http://fieldworks.sil.org/flex/)
{\it FileMaker Pro} (http://linguistics.berkeley.edu/~jcgood/bifocal/FileMakerRev.html) 

These non-web-based/stand-alone software have some great features such as auto-glossing built on machine learning (Toolbox, FLEx) 
and web-publishing capabilities (FileMaker Pro). However, Toolbox and FLEx seem to support only certain platform 
(either PC or Linax, but not both), limiting collaborating possibilities with Mac users. Since they are non-web-based, 
sharing 

FileMaker Pro is a general purpose database software and can be customized and programmed 
according to the needs of researchers. However, researchers will need to learn the software first. Many universities 
in fact hire a programmer to make it into a proper grammar building tool. 


\section {App Description}

Describe the project or program and provide information on how it will be implemented. Include 
information on what will be accomplished and the desired outcome. 

\section {Goals \& Objectives}

Describe the project objectives in measurable terms that address the academic and technological needs of language teachers, linguists, etc. (it's for linguists but other people can benefit from it as well i.e. help create dictionaries for endangered languages, which will benefit the communities of these languages; L2 acq. teachers).  

\section {Timeline}

How long does it take us to make it? Project progress over time. What we've done and what we'll do.

\section {Budget}

Include in the budget all expenses for the app, including necessary training costs. Mention any co-funding that you are using from other sources. You may want to include a brief narrative of expenses along with a table of individual cost components. 

\section {Evaluation}

Provide information on the metrics that will be used to determine the effectiveness of the project. It has to meet certain data management criteria -- any scientist can discover and use the data over time; it will benefit the scientific community. We are abiding by the best practice standards set by DataOne.   

\section {Staff \& Organizational Information}

Include the staff qualifications, certifications, and skills. Describe the organization and include 
information indicating the organization�s capacity to implement and sustain the program--i.e. info about iLanguage Lab LTD. CVs from people from the company, experience, etc.

Example format: 

\exg. n-tuop'ti-m\\
1-window-POSS.SG\\
`My window'


\end{document}