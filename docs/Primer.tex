\documentclass[12 pt]{article}
\usepackage{fullpage}
\usepackage{rotating}
\usepackage{tipa}
\usepackage{amsmath}
\usepackage{linguex}
\usepackage{enumerate}
\usepackage[sc]{mathpazo}
\linespread{1.05}         % Palatino needs more leading (space between lines)
\usepackage[T1]{fontenc}


\title{The FieldLinguists' App}
\author{}
\date{}

\begin{document}
\maketitle{}

\section {Project Abstract}

The FieldLinguists' App is an OpenSource database that allows language researchers to securely enter, store, organize, annotate, and share linguistic data. The application is accessible everywhere; it runs on three different systems (Mac, Linux, and Windows)  and is suitable for both online and offline use.  Furthermore, the application will be created with collaborative goals in mind;  data will be syncable and sharable with other researchers.  Researchers can form teams that contributes to a single corpus, where team members can modify and discuss the data from within the application. The system will also have a simple and friendly user interface, allowing participants to drag and drop files (audio, video, text), or record it directly into the database when using the Android app. In addition, the application will have import and export capabilities for multiple file types.  Most importantly, the application is designed intuitively and theory free, so it is not necessary to be a field linguist or programmer to figure out how it works.

\section {Statement of Need}

The statement of need should describe the problem that the project will attempt to address. Also, 
describe the population that will be served.  

\section {App Description}

The FieldLinguists' App will enable those interested in language research, preservation, and documentation to securely enter, store, organize, annotate, and share linguistic data. Moreover, the application will be easily customizable to fit specific needs. To accomplish these tasks, the database will be equipped with a variety of features. 
The following functional requirements are based on a few important considerations where most existing field linguistics/corpus linguistics databases applications fall short. 
\begin{itemize}

\newpage

\item {\bf Modern and Intuitive}

\begin{itemize}
\item { \bf Simple} The system will be designed to replace Word Documents or LaTeX documents which is a very common way fieldlinguists store data because it requires no training, doesn't require a complicated set-up for data categories, and takes no time to add new categories.  

\item {\bf Attractive} The system will have a modern design like many of the popular websites such as google and be customizable so that the user include a picture of where they are doing research as background.

\end{itemize}

\item {\bf Powerful}
\begin{itemize}
\item {\bf Smart} The application will guess what users do most often, and automate the process for them. Most importantly, the system will have automated predictable glossing information .
\item {\bf  Searchable} The application will be designed for search as this is one of the most fundamental tasks a language researcher must be able to do. The search will go far beyond traditional string matches and database indexes; it will be able to display data in context.
\end{itemize}

\item{\bf Data-Centric}
\begin{itemize}
\item {\bf Atheoretical} The application will not include categories or linguistic frameworks or theoretical constructs that must be tied to the data.  The application allows an analysis to develop organically as data collection proceeds as opposed to imposing a particular construct upon entry.  Researchers can set and change their own categories to the data whenever they choose to.
\item {\bf Collaborative} The system will have users and teams, and permissions for corpora. Permissions will ensure that data can be safely shared and edited by multiple users. Moreover, the corpus will be versioned so that users can track changes and revert mistakes.
\item {\bf Sharable} The application will allow researchers to share their data to anyone interested in their work.
\end{itemize}


\item{\bf Accessable}
\begin{itemize}
\item { \bf Cross-Platform} The application will run on Mac, Linux, and Windows computers. The application will be installable as a Chrome extension and available on any device that runs a browser.
\item  {\bf Portable} Touch tablets are one of the easiest tools to carry and use in field; they have a long battery life; they can play videos or show images for the consultant to elicit complicated contexts; and they permit recording audio and video and direct publishing to YouTube and/or other services. Furthermore, Android tablets are particularly easy to program and integrate the microphone directly into the database.
\item {\bf Work offine} Running a webapp offline will have considerable consequences for how data is stored, how data is retrieved, and how much data can be used while offline. Most browsers have limits on the amount of data a webapp can store offline. By delivering a version of the app in a Chrome Extension, which has permission to have unlimited storage, researchers will be able to have a significant portion of their data at their fingertips, regardless of the location.
\end{itemize}



\item{\bf Open}
\begin{itemize}
\item {\bf OpenData}. Corpora often contain sensitive information, informant stories and other information which must be kept confidential. Having confidential data in plain text in a corpus forces the entire corpus to be kept confidential. Instead, the system will encrypt confidential data and store the data in the corpus encrypted. To access the plain text the user will have to log in and use a password to decrypt the data. This design has important ramifications for exporting data, and for editing the data outside the application.
\item { \bf OpenSource}. Being OpenSource allows departments to install and customize the database application to tailor specific needs without worry that the company behind the software will disappear or stop maintaining the software. In addition, OpenSourcing the software on GitHub will allow linguists with scripting or programming experience to contribute back to the software to make it more customized to their needs, language typologies, or linguistics research areas.
\item {\bf Unicode}. Encoding problems and losing data should be behind us in the days of unicode. However, many existing fieldlinguistics databases were built in programming languages that did not support unicode, so the unicode support is dangerously fragile.
\end{itemize}

\end{itemize}





\section {Goals \& Objectives}

Describe the project objectives in measurable terms that address the academic and technological needs of language teachers, linguists, etc. (it's for linguists but other people can benefit from it as well i.e. help create dictionaries for endangered languages, which will benefit the communities of these languages; L2 acq. teachers).  

\section {Staff \& Organizational Information}

 iLanguage Lab is a Montreal based company that develops tools in the form of experimentation and data collection apps for Android and Chrome in collaboration with researchers at UdeM, UQAM, McGill and Concordia. Previous research applications includes the Bilingual Aphasia Test, AuBlog, OPrime and SpyOrNot.  Furthermore, iLanguage Lab has a background in assisting researchers obtain results and publications.  The AuBlog application was employed to investigate evolving information structure and audienceless vs. audience oriented prosodies and culminated in a poster presented at Experimental and Theoretical Advances in Prosody Conference. The Bilingual Aphasia Test led to a presentation at the  Academy of Aphasia 49th Annual Meeting on  Aphasia Assessment on Android: recording voice, eye-gaze and touch for the BAT and a publication in the Academy of Aphasia. 

\subsection{Gina Cook M.A.}

Gina Cook received her Masters in Field Linguistics \& DESS in Computer Science and has worked as a computational linguist for companies such as Nuance and Idelia.  She founded iLanguage Lab with a vision to develop computational tools to help researchers as opposed to consumers.  She is an active contributor to OpenSource projects on GitHub focusing on integrating existing OpenSource libraries for Speech Recognition, Natural Language Processing, Eye Gaze analysis and Acoustic analysis into Android tablet applications.

\subsubsection{Publications}
\begin{itemize}

\item "Aphasia Assessment on Android: recording voice, eye-gaze and touch for the BAT." (with A. Marquis \& A. Achim). Poster at Academy of Aphasia 49th Annual Meeting, Mont\'eal, Qu\'ebec. October 2011.
\item "Eliciting evolving information structure and audienceless vs. audience oriented prosodies: experimentation on Android tablets." (with S. Kattoju). Poster at ETAP2 � Experimental and Theoretical Advances in Prosody, Montr\'eal, Qu\'ebec. September 2011.
\item "PDFtoAudioBook Android app" (Java, XML).Canadian University Software Engineering Conference (CUSEC) DemoCamp, Montr\'eal, Qu\'ebec. January 2011.
\item "Word features and word concatenation."Sixth Interdisciplinary Graduate Student Research Symposium, McGill University, Montr\'eal, Qu\'ebec. March 2009.
\item "The Structure of Long Distance Agreement in Hindi/Urdu." Invited Lecture in Advanced Syntax, Concordia University, Montr\'eal, Qu\'ebec. November 2007.
\item "The Phonological/Phonetic status of Productive Palatalization in Romanian." (with L. Spinu). Presented at the Seoul International Conference on Linguistics, Seoul National University, Seoul, South Korea. July 2006.
\end{itemize}


\subsection{M.E. Cathcart Ph.D.}

M.E. Cathcart completed her PhD at the University of Delaware with a dissertation grant funded by the National Science Foundation (NSF) for her field work in Cusco, Peru on Quechua. In addition, she also has a background of coursework in computational linguistics, at the University of Delaware and at the Linguistic Society of America�s Summer Institute.

\subsubsection{Publications}

\begin{itemize}
\item "Affected Arguments Cross-linguistically." S. Bosse, B. Bruening, M.E. Cathcart, A. E. Peng, M. Yamada. In: Tadic, M. Dimitrova-Vulchanova, M., Koeva, S. (eds.): FASSBL 6 The Sixth International Conference on Formal Approaches to South Slavic and Balkan Languages. 2008 (Proceedings) pp. 41-47.
\item "A New Grammatical Category: Impulsatives." Penn Linguistics Colloquium, Philadelphia March 2010
?Eliciting data for dissertation on Impulsatives: functional morpheme in context
\item"The Syntax and Semantics of Desideratives in Albanian." Georgetown Linguistics Society, Washington, D.C. February 2010
\item "Bi-Eventivity \& Affecting Arguments." S. Bosse, B. Bruening, M.E. Cathcart, H.-j. Cheng, A. E. Peng, M. Yamada. Formal Approaches to South Slavic and Balkan Languages, Dubrovnik (Croatia), 25-28 September 2008.
\item "Bi-Eventive Affect." S. Bosse, B. Bruening, M.E. Cathcart, H.-j. Cheng, A. E. Peng, M. Yamada.TEAL, Potsdam (Germany), 10-11 September 2008.
\end{itemize}


\subsection{Theresa Deering M.A.}

Theresa Deering has a  Bachelor's in Computer Science from Malaspina and a Master's in Computer Science from McGill University. Her thesis focused on the Least-Used Direction pivot rule for the Simplex Method of solving linear programs.

\begin{itemize}
\item "The Least-Used Direction Pivot Rule on Acyclic Unique Sink Orientations." Master's Thesis. McGill University, Montr\'eal, Qu\'ebec. July 2010.
\item "Worst-case Behaviour of History Based Pivot Rules on Acyclic Unique Sink Orientations of Hypercubes." Y. Aoshima, D. Avis, T. Deering, Y. Matsumoto, S. Moriyama. Submitted to AAAC. October 2011.
\end{itemize}

\section {Budget}

The FieldLinguists' App is composed of eight modules and thus the cost is divided into eight major components. In addition, a separate price is given for software architecture and for 1 year of user support and project growth; needed to make a longterm viable and useful tool that fieldlinguists can adopt for their labs or for their field methods courses. The collaboration module is used to permit collaboration with teams and users. The Corpus Module is used to sync, share, edit, tag, categorize and open data. The Lexicon Module is used to house, and read lexicon entries to be used for the glosser.  The Dictionary Module is used to share the lexicon in the form of a WordNet/Wiktionary dictionary with the language community as required by some grants. The Glosser Module is used to automatically gloss datum, smarter than the standard lexicon. Finally, the Aligner Module is used to create TextGrids from the orthography and the audio files, used for prosody and phonetic analysis. 


The prices given include a 40\% Researcher discount and a 10\% Open Source discount.  An additional 40\% discount is given for certain components that can be made with the help of students from Concordia University, Hisako Noguchi and Yulia Manyakina.  The cost is calculated by determining the time in hours and multiplying by \$48, the average rate of a software developer in Montr\'eal.  


\begin{table}[htbp]
\begin{center}
  \begin{tabular}{ | cc | }
\hline
    Module & Price \\ 
\hline
    Software Architecture & \$1,555.20  \\ 
    Collaboration Module & \$5,728.32  \\ 
    Corpus Module & \$9,201.60 \\
     Web Spider Module & \$2,177.28\\
     Phonological Search Module & \$2,177.28 \\
    Lexicon Module & \$7,340.54 \\ 
    Dictionary Module & \$18,781.63 \\
        Glosser Module & \$17,770.75 \\
Aligner Module & \$ 9,787.39\\ 
User Support & \$30,246.70 \\
TVS and TPQ & \$10,833.32 \\
Total &  \$83,176.04 \\

\hline
  \end{tabular}
  \caption{Cost Summary}
  \label{tab:label}
  \end{center}
\end{table}



\newpage
\subsection{Collaboration Module}

\footnotesize
\begin{sidewaystable}[htbp]
  \begin{tabular}{ | cccccccc | }
\hline

Collaboration Module&	&	&	&	&	\\ 
\hline
Collaboration Module: used to permit collaboration, teams and users&	&	&	&	&	&	&	\\ 
Iteration&	Hours&	Technology&	Rate&	Cost Closed Source &	Research Reduction 40\%&	Reduction Open Source 10\%&	 Reduction Student Made  \%40\\ 
Software Architecture Design&	20&	Software Engineering&	48&	960&	576&	518.4&	311.04\\ 
Collaboration API on central server&	30&	Software Engineering&	48&	1440&	864&	777.6&	777.6\\ 
Users Model&	15&	Javascript&	48&	720&	432&	388.8&	233.28\\ 
Informant Model&	15&	Javascript&	48&	720&	432&	388.8&	233.28\\ 
Team Model&	15&	Javascript&	48&	720&	432&	388.8&	233.28\\ 
Bot Model&	15&	Javascript&	48&	720&	432&	388.8&	233.28\\ 
User Activity Model&	8&	Javascript&	48&	384&	230.4&	207.36&	124.416\\ 
Team Feed Widget&	25&	HTML5&	48&	1200&	720&	648&	648\\ 
User list item Widget&	16&	HTML5&	48&	768&	460.8&	414.72&	248.832\\ 
Team Preferences Widget&	8&	HTML5&	48&	384&	230.4&	207.36&	124.416\\ 
User Profile Widget&	8&	HTML5&	48&	384&	230.4&	207.36&	124.416\\ 
User Tests&	30&	Javascript&	48&	1440&	864&	777.6&	466.56\\ 
Informant Tests&	30&	Javascript&	48&	1440&	864&	777.6&	466.56\\ 
Team Tests&	30&	Javascript&	48&	1440&	864&	777.6&	466.56\\ 
Android Deployment&	15&	Java&	48&	720&	432&	388.8&	388.8\\ 
Chrome Extension Deployment&	20&	Javascript&	48&	960&	576&	518.4&	518.4\\ 
Heroku Deployment&	5&	Integration&	48&	240&	144&	129.6&	129.6\\ 
\# of weeks with 3 full time personnel&	2.5416666667&	&	&	14640&	8784&	7905.6&	\$5,728.32\\ 
\hline
  \end{tabular}
  \caption{Cost Summary}
  \label{tab:label}
\end{sidewaystable}
%%Include in the budget all expenses for the app, including necessary training costs. Mention any co-funding that you are using from other sources. You may want to include a brief narrative of expenses along with a table of individual cost components. 






%\section {Timeline}
%
%How long does it take us to make it? Project progress over time. What we've done and what we'll do.
%
%
%
%\section {Evaluation}
%
%Provide information on the metrics that will be used to determine the effectiveness of the project. It has to meet certain data management criteria -- any scientist can discover and use the data over time; it will benefit the scientific community. We are abiding by the best practice standards set by DataOne.   
%
%Example format: 
%
%\exg. n-tuop'ti-m\\
%1-window-POSS.SG\\
%`My window'
%
%
\end{document}