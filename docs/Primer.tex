\documentclass[12 pt]{article}
\usepackage{fullpage}
\usepackage{tipa}
\usepackage{amsmath}
\usepackage{linguex}
\usepackage{enumerate}
\usepackage[sc]{mathpazo}
\linespread{1.05}         % Palatino needs more leading (space between lines)
\usepackage[T1]{fontenc}


\title{The Field Linguists' App}
\author{}
\date{}

\begin{document}
\maketitle{}

\section {Project Abstract}

The FieldLinguists' App is an OpenSource database that allows language researchers to securely enter, store, organize, annotate, and share linguistic data. The application is accessible everywhere; it runs on three different systems (Mac, Linux, and Windows)  and is suitable for both online and offline use.  Furthermore, the application will be created with collaborative goals in mind;  data will be syncable and sharable with other researchers.  Researchers can form teams that contributes to a single corpus, where team members can modify and discuss the data from within the application. The system will also have a simple and friendly user interface, allowing participants to drag and drop files (audio, video, text), or record it directly into the database when using the Android app. In addition, the application will have import and export capabilities for multiple file types.  Most importantly, the application is designed intuitively and theory free, so it is not necessary to be a field linguist or programmer to figure out how it works.

\section {Statement of Need}

The statement of need should describe the problem that the project will attempt to address. Also, 
describe the population that will be served.  

\section {App Description}

Describe the project or program and provide information on how it will be implemented. Include 
information on what will be accomplished and the desired outcome. 

\section {Goals \& Objectives}

Describe the project objectives in measurable terms that address the academic and technological needs of language teachers, linguists, etc. (it's for linguists but other people can benefit from it as well i.e. help create dictionaries for endangered languages, which will benefit the communities of these languages; L2 acq. teachers).  

\section {Staff \& Organizational Information}

 iLanguage Lab is a Montreal based company that develops tools in the form of experimentation and data collection apps for Android and Chrome in collaboration with researchers at UdeM, UQAM, McGill and Concordia. iLanguage Lab has a background in assisting researchers obtain results by creating applications that suit the researcher's needs.  Previous research applications includes the Bilingual Aphasia Test, AuBlog, and OPrime. The Bilingual Aphasia Test led to  a presentation at the  Academy of Aphasia 49th Annual Meeting on  Aphasia Assessment on Android: recording voice, eye-gaze and touch for the BAT and a publication in the Academy of Aphasia.  The AuBlog application was employed to investigate evolving information structure and audienceless vs. audience oriented prosodies and culminated in a poster presented at Experimental and Theoretical Advances in Prosody Conference. 

\subsection{Gina Cook M.A.}

Gina Cook received her Masters in Field Linguistics \& DESS in Computer Science. 





\section {Budget}

Include in the budget all expenses for the app, including necessary training costs. Mention any co-funding that you are using from other sources. You may want to include a brief narrative of expenses along with a table of individual cost components. 

\section {Timeline}

How long does it take us to make it? Project progress over time. What we've done and what we'll do.



\section {Evaluation}

Provide information on the metrics that will be used to determine the effectiveness of the project. It has to meet certain data management criteria -- any scientist can discover and use the data over time; it will benefit the scientific community. We are abiding by the best practice standards set by DataOne.   

Example format: 

\exg. n-tuop'ti-m\\
1-window-POSS.SG\\
`My window'


\end{document}