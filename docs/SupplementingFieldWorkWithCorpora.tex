
\documentclass[12pt]{article}
%\usepackage{fullpage}
\usepackage{rotating}
\usepackage{tipa}
\usepackage{amsmath}
%\usepackage{linguex}
\usepackage{enumerate}

\usepackage[T1]{fontenc}

\hyphenpenalty= 7000 

\title{Supplementing field work with corpora:\\ \vspace{2 mm} {\large An investigation of ergativity and anti-passives in Inuktitut\\ Ling 488 Indepedent Study 1}}
\author{Louisa Bielig}
\date{}

\begin{document}
\maketitle{} 

\tableofcontents

\section {Project Abstract}

\appendix

\section{Corpus Creation}





\begin{figure}
\begin{verbatim}

Sample Output:
--------------

The end result is an html file which automatically makes an alligned corpus and appends it to the top of the document in three formats.  If you want another format, you can modify the alignChaptersAndVerses.js script.
\end{verbatim}
\end{figure}

\begin{figure}
\begin{verbatim}
* Raw Aligned Text

<pre>

1co:9:6 ᐅᕝᕙᓘ ᐸᕐᓇᐹᓯᓗ ᐅᕙᒍᒃ ᐃᓅᑦᔪᑎᒋᓂᐊᖅᑕᑦᑎᓐᓂᒃ ᐃᖅᑲᓁᔭᖅᑐᑑᔭᕆᐊᖃᖅᐱᓅᒃ? 
1co:9:6 ¿Wa ca tucultique'ex chéen teen yéetel Bernabé unaj c meyaj yéetel áakab? 
1co:9:6 Eller hafver jag och Barnabas allena icke magt sammaledes göra? 
1co:9:6 Or is it only Barnabas and I who have to work to support ourselves?

1co:9:7 ᓇᓪᓕᐊᑦ ᐅᓇᑕᖅᑐᒃᓴᐅᓪᓗᓂ ᐊᑐᕐᖕᓂᐊᖅᑕᒥᓂᒃ ᓇᖕᒥᓂᖅ ᐊᑭᓖᓲᖑᕚ? ᓇᓪᓕᐊᑦ ᕔᓂᒃᓴᓂᒃ ᑲᓐᖓᖅᓱᓚᐅᖅᑕᒥᓂᒃ ᐱᕈᖅᓰᕕᖁᑎᒥᓂᑦ ᐱᕈᖅᑐᓂᒃ ᓂᕆᕙᓐᖏᓛᖅ? ᓇᓪᓕᐊᓪᓗ ᐆᒪᔪᓂᒃ ᑲᒪᔨᐅᔪᖅ ᐆᒪᔪᖁᑎᒥ ᐃᒻᒧᖏᓐᓂᒃ ᐃᒻᒧᒃᑖᖅᕕᖃᖅᐸᓐᖏᓛᖅ?
1co:9:7 ¿Máax cu beetic u soldadoil yéetel cu tojoltic ti' xan ba'ax cu xupic? ¿Máax cu pakic uva cu dzo'ocole' ma' tu jaantic u yich? ¿Máax cu canantic j tamano'ob cu dzo'ocole' ma' tu yukik u kaab u yiim le j tamano'obo'? 
1co:9:7 Ho tjenar till krig på sin egen sold någon tid? Ho planterar en vingård, och icke äter af hans frukt? Eller ho vaktar en hjord, och äter icke af hjordsens mjölk? 
1co:9:7 What soldier has to pay his own expenses? What farmer plants a vineyard and doesn’t have the right to eat some of its fruit? What shepherd cares for a flock of sheep and isn’t allowed to drink some of the milk? 

\end{verbatim}
\end{figure}

\begin{figure}
\begin{verbatim}


</pre>

* XML

```xml

<?xml version="1.0" encoding="UTF-8"?>
<xml>
   <book book="1co">
      <chapters>
         <chapter9 chapterNumber="9">
            <verses>
               <verse6 verseNumber="6">
                  <inuktitut>ᐅᕝᕙᓘ ᐸᕐᓇᐹᓯᓗ ᐅᕙᒍᒃ ᐃᓅᑦᔪᑎᒋᓂᐊᖅᑕᑦᑎᓐᓂᒃ ᐃᖅᑲᓁᔭᖅᑐᑑᔭᕆᐊᖃᖅᐱᓅᒃ?</inuktitut>
                  <yucatec>¿Wa ca tucultique'ex chéen teen yéetel Bernabé unaj c meyaj yéetel áakab?</yucatec>
                  <swedish>Eller hafver jag och Barnabas allena icke magt sammaledes göra?</swedish>
                  <english>Or is it only Barnabas and I who have to work to support ourselves?</english>
               </verse6>
               <verse7 verseNumber="7">
                  <inuktitut>ᓇᓪᓕᐊᑦ ᐅᓇᑕᖅᑐᒃᓴᐅᓪᓗᓂ ᐊᑐᕐᖕᓂᐊᖅᑕᒥᓂᒃ ᓇᖕᒥᓂᖅ ᐊᑭᓖᓲᖑᕚ? ᓇᓪᓕᐊᑦ ᕔᓂᒃᓴᓂᒃ ᑲᓐᖓᖅᓱᓚᐅᖅᑕᒥᓂᒃ ᐱᕈᖅᓰᕕᖁᑎᒥᓂᑦ ᐱᕈᖅᑐᓂᒃ ᓂᕆᕙᓐᖏᓛᖅ? ᓇᓪᓕᐊᓪᓗ ᐆᒪᔪᓂᒃ ᑲᒪᔨᐅᔪᖅ ᐆᒪᔪᖁᑎᒥ ᐃᒻᒧᖏᓐᓂᒃ ᐃᒻᒧᒃᑖᖅᕕᖃᖅᐸᓐᖏᓛᖅ?</inuktitut>
                  <yucatec>¿Máax cu beetic u soldadoil yéetel cu tojoltic ti' xan ba'ax cu xupic? ¿Máax cu pakic uva cu dzo'ocole' ma' tu jaantic u yich? ¿Máax cu canantic j tamano'ob cu dzo'ocole' ma' tu yukik u kaab u yiim le j tamano'obo'?</yucatec>
                  <swedish>Ho tjenar till krig på sin egen sold någon tid? Ho planterar en vingård, och icke äter af hans frukt? Eller ho vaktar en hjord, och äter icke af hjordsens mjölk?</swedish>
                  <english>What soldier has to pay his own expenses? What farmer plants a vineyard and doesn’t have the right to eat some of its fruit? What shepherd cares for a flock of sheep and isn’t allowed to drink some of the milk?</english>
               </verse7>
            </verses>
         </chapter9>
      </chapters>
   </book>
</xml>

\end{verbatim}
\end{figure}

\begin{figure}
\begin{verbatim}


* JSON

```json
{
   "book":{
      "_book":"1co",
      "chapters":{
         "chapter9":{
            "_chapterNumber":"9",
            "verses":{
               "verse6":{
                  "_verseNumber":"6",
                  "inuktitut":"ᐅᕝᕙᓘ ᐸᕐᓇᐹᓯᓗ ᐅᕙᒍᒃ ᐃᓅᑦᔪᑎᒋᓂᐊᖅᑕᑦᑎᓐᓂᒃ ᐃᖅᑲᓁᔭᖅᑐᑑᔭᕆᐊᖃᖅᐱᓅᒃ? ",
                  "yucatec":"¿Wa ca tucultique'ex chéen teen yéetel Bernabé unaj c meyaj yéetel áakab? ",
                  "swedish":"Eller hafver jag och Barnabas allena icke magt sammaledes göra? ",
                  "english":"Or is it only Barnabas and I who have to work to support ourselves?"
               },
               "verse7":{
                  "_verseNumber":"7",
                  "inuktitut":"ᓇᓪᓕᐊᑦ ᐅᓇᑕᖅᑐᒃᓴᐅᓪᓗᓂ ᐊᑐᕐᖕᓂᐊᖅᑕᒥᓂᒃ ᓇᖕᒥᓂᖅ ᐊᑭᓖᓲᖑᕚ? ᓇᓪᓕᐊᑦ ᕔᓂᒃᓴᓂᒃ ᑲᓐᖓᖅᓱᓚᐅᖅᑕᒥᓂᒃ ᐱᕈᖅᓰᕕᖁᑎᒥᓂᑦ ᐱᕈᖅᑐᓂᒃ ᓂᕆᕙᓐᖏᓛᖅ? ᓇᓪᓕᐊᓪᓗ ᐆᒪᔪᓂᒃ ᑲᒪᔨᐅᔪᖅ ᐆᒪᔪᖁᑎᒥ ᐃᒻᒧᖏᓐᓂᒃ ᐃᒻᒧᒃᑖᖅᕕᖃᖅᐸᓐᖏᓛᖅ?",
                  "yucatec":"¿Máax cu beetic u soldadoil yéetel cu tojoltic ti' xan ba'ax cu xupic? ¿Máax cu pakic uva cu dzo'ocole' ma' tu jaantic u yich? ¿Máax cu canantic j tamano'ob cu dzo'ocole' ma' tu yukik u kaab u yiim le j tamano'obo'? ",
                  "swedish":"Ho tjenar till krig på sin egen sold någon tid? Ho planterar en vingård, och icke äter af hans frukt? Eller ho vaktar en hjord, och äter icke af hjordsens mjölk? ",
                  "english":"What soldier has to pay his own expenses? What farmer plants a vineyard and doesn’t have the right to eat some of its fruit? What shepherd cares for a flock of sheep and isn’t allowed to drink some of the milk? "
               }
            }
         }
      }
   }
}
```

\end{verbatim}
\end{figure}

\begin{figure}
\begin{verbatim}

Install:
--------

1. Download [Node.js](http://nodejs.org/) if you don't already have it
2. Then download this project

```bash
$ wget https://github.com/louisa-bielig/MultilingualCorporaExtractor/archive/master.zip
$ unzip master.zip
$ cd MultilingualCorporaExtractor
$ npm install 
```
\end{verbatim}
\end{figure}

\begin{figure}
\begin{verbatim}


Usage: 
------
Here is a sample use of the interactive script:

<pre>

$ ./createdata.sh
Enter the three character code for the book you want to use for your corpus
e.g. gen for Genesis and press [ENTER]: 1co
Enter the starting chapter number for 1co and press [ENTER]: 9
Enter the ending chapter number for 1co and press [ENTER]: 9

Enter the language number code and press [ENTER]: 455
Enter the language text code and press [ENTER]: inuktitut
Working...
9 9 1co 455 inuktitut 1co-9-9-1370897837.html
Chapter 9 downloaded.
Finished!

Enter the language number code and press [ENTER]: 455
Enter the language text code and press [ENTER]: inuktitut
Working...
9 9 1co 455 inuktitut 1co-9-9-1370897837.html
Chapter 9 downloaded.
Finished!

Enter the language number code and press [ENTER]: 324
Enter the language text code and press [ENTER]: yucatec
Working...
9 9 1co 324 yucatec 1co-9-9-1370897837.html
Chapter 9 downloaded.
Finished!

Enter the language number code and press [ENTER]: 161
Enter the language text code and press [ENTER]: swedish
Working...
9 9 1co 161 swedish 1co-9-9-1370897938.html
Chapter 9 downloaded.
Finished!

Enter the language number code and press [ENTER]: 116
Enter the language text code and press [ENTER]: english
Working...
9 9 1co 116 english 1co-9-9-1370897938.html
Chapter 9 downloaded.
Finished!

Enter the language number code and press [ENTER]: exit
$ google-chrome 1co-9-9-1370897938.html &
#!/bin/bash

</pre>



License:
--------

Apache 2.0 

\end{verbatim}
\end{figure}
\end{document}
\begin{figure}
	
\begin{verbatim}
////
//
// Javascript file to align and display information downloaded
// from youversion.com with our creatadata.sh and download.sh scripts.
//
////


var book = document.body.getAttribute('id');
var chapters = {};

// check if a value is a number
function isNumber(n) {
  return !isNaN(parseFloat(n)) && isFinite(n);
}

// main function that starts with the entire document
// and finds each language in the document
$('body').find('.language').each(function() {

  var languagecode = this.getAttribute('id');

  // verify if all chapters for each language have the same number of verses
  $(this).children('.chapter').each(function() {
    try {
     var chapter = $(this).children('.label')[0].innerHTML;
   } catch (err) {
    alert('This chapter did not download correctly.');
  }

  // cycle through each chapter to find all of the verses.
  // For each of these verses, save the number and the text for this verse
  // and creates aligned verse groups.
  (function(chapterNum, chapterDiv, langCode) {
    chapterNumber = 'chapter' + chapterNum;
    chapters[chapterNumber] = chapters[chapterNumber] || {};
    chapters[chapterNumber]._chapterNumber = chapterNum;
    chapters[chapterNumber].verses = chapters[chapterNumber].verses || {};

    $(chapterDiv).find('.verse').each(function() {

      try {
        var verseNum = $(this).find('.label')[0].innerHTML;
        verseNumber = 'verse' + verseNum;
        console.log('working: ' , this);
        chapters[chapterNumber].verses[verseNumber] = chapters[chapterNumber].verses[verseNumber] || {};
        chapters[chapterNumber].verses[verseNumber]._verseNumber = verseNum;
        chapters[chapterNumber].verses[verseNumber][langCode] = $(this).find('.content').html().replace(/\n/g,"").replace(/  */g," ");
      } catch (err) {
        console.log('verse not working: ' , this);
      }
    });
  })(chapter, this, languagecode);

});

});

// Now that our "chapters" object is created and contains all of our aligned data, 
// we will cycle through it to display the text information for the user in the browser
var asRawText = '';
for (chapter in chapters) {
  for (verse in chapters[chapter].verses) {
    var metadata = book + ':' + chapter.replace("chapter","") + ':' + chapters[chapter].verses[verse]._verseNumber;
    for (language in chapters[chapter].verses[verse]) {
      if (isNumber(chapters[chapter].verses[verse][language])) {
        continue;
      } else {
        var thisline = metadata + ' ' + chapters[chapter].verses[verse][language] + '\n';
        console.log(thisline);
        asRawText += thisline;
      }
    }
    asRawText += '\n';
  }
}

// For useful data export, we will also convert our "chapters" object into XML and JSON.
var X2JS = new X2JS();
var finalJSON = '{"book" : {"_book":"' + book + '", "chapters":' + JSON.stringify(chapters) + '}}';
var xmlDocStr = X2JS.json2xml_str(JSON.parse(finalJSON));

$('body').prepend('<label class="json-label">JSON</label><textarea class="json">' + finalJSON + '</textarea>');
$('body').prepend('<label class="xml-label">XML</label><textarea class="xml"><?xml version="1.0" encoding="UTF-8"?><xml>' + xmlDocStr + '</xml></textarea>');
$('body').prepend('<label class="rawtext-label">Text</label><textarea class="rawtext">' + asRawText + '</textarea>');

\end{verbatim}

	\caption{Corpus2morphology calls all the other scripts in the correct order with the correct argument so that another user can see how the algorithm is implemented.}
	\label{fig:corpus2morphology}
\end{figure}






\end{document}


